\documentclass{article}
\usepackage{graphicx} % Required for inserting images
\usepackage[spanish]{babel}
\usepackage[utf8]{inputenc}
\title{Hipótesis del bosque oscuro}
\author{Isaac Villa}
\date{1 de abril de 2024}

\begin{document}

\maketitle

\section{Introducción}

``La Hipótesis del Bosque Oscuro'' es un video de divulgación literaria y científica publicado por el usuario de Youtube ``El Robot de Platón''. El video explica la perspectiva desde la que la novela ``El bosque oscuro'' (Liu Cixin, 2008) aborda la paradoja de Fermi, planteando la posibilidad de que civilizaciones extraterrestres avanzadas no solo se escondan para protegerse, sino que también puedan adoptar medidas agresivas dadas a eliminar cualquier amenaza percibida. Esta posible solución a la paradoja de Fermi tiene un precedente en la hipótesis homónima de la novela: la hipótesis del bosque oscuro, ideada por el astrónomo David Brin. El autor del video también compara ``El bosque oscuro'' con otra novela de ciencia ficción que también versa sobre la paradoja de Fermi: ``The killing star'' (Charles Pellegrino y George Zebrowski).

\maketitle

\section{La hipótesis del bosque oscuro}

La Hipótesis del Bosque Oscuro es una teoría especulativa que busca explicar la aparente ausencia de detección de civilizaciones extraterrestres a pesar de la alta probabilidad de que existan en el universo. Propuesta como respuesta a la Paradoja de Fermi, plantea que las civilizaciones avanzadas podrían optar por ocultarse deliberadamente para evitar ser detectadas por otras potenciales civilizaciones, que podrían representar una amenaza.

\maketitle

\section{``The killing star''}
``The Killing Star'' es una novela de ciencia ficción escrita por Charles R. Pellegrino y George Zebrowski. Ambientada en un futuro distante, la historia explora la Paradoja de Fermi desde una perspectiva inquietante y sombría. En el universo de la novela, las civilizaciones extraterrestres no solo optan por ocultarse, sino que también pueden considerar cualquier otra forma de vida inteligente como una amenaza potencial.



\end{document}
